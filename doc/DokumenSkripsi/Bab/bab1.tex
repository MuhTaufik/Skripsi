\chapter{Pendahuluan}
\label{chap:Pendahuluan}
   
\section{Latar Belakang}
\label{sec:Latar Belakang}

Kurikulum menjadi komponen acuan oleh setiap satuan pendidikan. Kurikulum berkembang sejalan dengan perkembangan teori dan praktek pendidikan, selain itu juga bervariasi sesuai dengan aliran atau teori pendidikan yang dianut pemangku kebijakan. Kurikulum memiliki kedudukan yang sangat sentral dalam keseluruhan proses pendidikan. Kurikulum juga mengarahkan segala bentuk aktivitas pendidikan kepada tercapainya tujuan-tujuan pendidikan. Sehingga kurikulum menjadi elemen pokok dalam sebuah layanan program pendidikan. Kurikulum juga memiliki peranan penting dalam pendidikan, kaitannya yaitu dengan penentuan arah, isi, dan proses pendidikan yang pada akhirnya menentukan macam dan kualifikasi lulusan suatu lembaga pendidikan. Dengan kata lain kurikulum menjadi syarat mutlak dari pendidikan dan kurikulum merupakan bagian yang tak terpisahkan dari pendidikan dan pengajaran. Sehingga sangatlah sulit dibayangkan bagaimana bentuk pelaksanaan suatu pendidikan tanpa adanya kurikulum.

Pada dasarnya kurikulum tidak hanya berisikan tentang petunjuk teknis materi pembelajaran. Kurikulum merupakan sebuah program terencana dan menyeluruh, yang secara tidak langsung menggambarkan manajemen pendidikan suatu bangsa. Dengan begitu otomatis kurikulum memegang peran yang sangat penting dan strategis dalam kemajuan dunia pendidikan suatu negara.

Efektifitas implementasi dan pengembangan kurikulum di lapangan sangatlah bergantung pada kompetensi sumber daya yang tersedia di universitas untuk memfasilitasi pengajar dalam mengartikulasi topik-topik yang termuat dalam kurikulum. Pengajar yang menjalankan segala sesuatu yang terjadi dalam kelasnya. Sehingga keberhasilan pengembangan kurikulum juga bergantung pada manajemen dari setiap pengajar. Kurikulum sendiri pada setiap satuan pendidikan sebagai alat penggerak pendidikan. Dengan kesesuaian dan ketepatan setiap komponen yang ada dalam kurikulum diharapkan sasaran dan tujuan pendidikan akan tercapai secara maksimal.

Dikarenakan peran kurikulum sendiri sangatlah penting dalam upaya pencapaian tujuan pendidikan nasional, maka pemerintah Indonesia telah melakukan berbagai macam upaya untuk merevisi, mengembangkan dan menyempurnakan desain kurikulum pendidikan nasional Indonesia untuk bisa menghasilkan proses dan produk pendidikan yang bermutu dan kompetitif. Kurikulum tidak bersifat statis, sehingga munculnya kurikulum disesuaikan dengan perkembangan zaman dan tuntutan kemajuan kehidupan dalam masyarakat. Kurikulum memang selalu berkembang dan menyelaraskan diri dengan kemajuan zaman. Untuk itu  pengembangan kurikulum berupa proses yang dinamis dan integratif yang memang perlu diupayakan melalui langkah-langkah yang sistematis, profesional dan melibatkan seluruh aspek yang terkait dalam tercapainya tujuan pendidikan nasional. Namun jika kita melihat di lapangan perubahan kurikulum yang dirasa menjadi suatu siklus yang ekstrem malah menunjukkan banyak masalah karena perubahan kurikulum itu sendiri yang terlalu sering. Setiap pergantian kepemimpinan atau perubahan menteri pendidikan sendiri hampir bisa dipastikan akan terjadi perubahan kurikulum yang akhirnya membuat para aktor di bidang pendidikan mendapat kurikulum yang tidak konsisten. 



\section{Rumusan Masalah}
\label{sec:rumusan masalah}

Berdasarkan latar belakang masalah yang telah dijelaskan, rumusan masalah pada penelitian ini adalah:
\begin{enumerate}
\item Tujuan diubahnya kurikulum lama (2013) ke kurikulum baru (2018).
\item Alasan diubahnya kurikulum lama (2013) ke kurikulum baru (2018).
\item Dampak yang terjadi dengan danya perubahan kurikulum. 

\end{enumerate}

\section{Tujuan}
\label{sec:tujuan}
Berdasarkan rumusan masalah di atas, maka tujuan dari penelitian ini adalah: 
\begin{enumerate}
\item Mahasiswa mengetahui tentang tujuan perubahan kurikulum.
\item Mahasiswa mengetahui tentang alasan diubahnya kurikulum lama ke kurikulum baru.
\item Mahasiswa mengetahui tentang dampak perubahan kurikulum.
\end{enumerate}

\section{Batasan Masalah}
\label{sec:batasan}
Adapun batasan masalah yang didapat dari tujuan dan rumusan masalah di atas adalah:

\begin{enumerate}
\item Perangkat lunak yang dikemangkan akan berbasis web.
\item Tampilan pada web hanya dibuat dalam bentuk graf.
\item Perangkat lunak yang dikembangkan hanya berlaku untuk mahasiswa Teknologi Informatika.
\end{enumerate}

\section{Metodologi Penelitian}
\label{sec:Metodologi Penelitian}
Dalam penyusunan skripsi ini mengikuti langkah-langkah metodologi penelitian sebagai berikut:

\begin{enumerate}
\item Melakukan studi pustaka untuk dijadikan referensi dalam pembangunan perangkat lunak pohon kurikulum.
\item Melakukan studi tentang penggunaan vis.js untuk visualisasi pohon kurikulum.
\item Melakukan studi tentang cara penggunaan DOT \textit{Language}
\end{enumerate}

\section{Sistematika Penulisan}
\label{sec:Sistematika Penulisan}
Keseluruhan bab yang disusun dalam penelitian ini terbagi kedalam bab-bab sebagai berikut:
\begin{enumerate}
\item Bab 1 Pendahuluan
Bab ini membahas mengenai latar belakang, rumusan masalah, tujuan, batasan masalah,
metodologi penelitian dan sistematika penulisan.
\item Bab 2 Dasar Teori
Bab ini membahas mengenai pengertian graf, apa itu DOT \textit{Language}, dan visualisasi menggunakan viz.js.
\item Bab 3 Analisis 

\item Bab 4 Perancangan 
Bab ini akan membahas mengenai perancangan struktur pohon kurikulum untuk mahasiswa, di mana nanti di dalamnya akan berisi mata kuliah, syarat tempuh, dan syarat lulus. 
\item Bab 5 Implementasi dan Pengujian
Bab ini akan membahas mengenai pengujian, implementasi kode program untuk membuat pohon kurikulum.
\item Bab 6 Kesimpulan dan Saran
Bab ini akan membahas mengenai kesimpulan dari penelitian yang telah dilakukan dan
saran-saran untuk pengembangan lebih lanjut dari penelitian ini.
\end{enumerate}