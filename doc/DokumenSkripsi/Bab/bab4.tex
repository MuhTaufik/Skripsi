\chapter{Perancangan}
\label{chap: Perancangan}

Pada bab ini akan dibahas mengenai perancangan perangkat lunak yang diimplementasi pada pohon kurikulum.

\section{Kebutuhan \textit{Input} dan \textit{Output}}
\label{sec: Kebutuhan Input dan Output}
Perancangan perangkat lunak pohon kurikulum dengan men \textit{generate} dari JSON ke \textit{DOT}. Input perangkat lunak merupakan kode, nama, prasyarat, sks, semester, dan wajib.
Kebutuhan input dan output perangkat lunak:
\begin{itemize}
\item \textit{Input} \\
Kebutuhan \textit{input} pada pohon kurikulum adalah
\begin{enumerate}
\item \textbf{kode}, berisikan kode mata kuliah
\item \textbf{nama}, berisikan nama mata kuliah
\item \textbf{sks}, memberitahukan kepada mahasiswa mata kuliah yang akan diambil memiliki beban berapa banyak.
\end{enumerate}
\item \textit{Output}\\
Output dari perangkat lunak adalah pohon kurikulum.
\end{itemize}

\section{Perancangan Kebutuhan Perangkat Lunak}
\label{sec: Perancangan Kebutuhan Perangkat Lunak}




\section{Perancangan Antarmuka}
\label{sec: Perancangan Antarmuka}