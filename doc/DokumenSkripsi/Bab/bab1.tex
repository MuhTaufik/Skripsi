\chapter{Pendahuluan}
\label{chap:Pendahuluan}
   
\section{Latar Belakang}
\label{sec:Latar Belakang}
Kurikulum didefinisikan sebagai seperangkat rencana dan pengaturan mengenai capaian pembelajaran 
lulusan, bahan kajian, proses, dan penilaian yang digunakan sebagai pedoman penyelenggaraan program studi menjadi sarana utama untuk mencapai tujuan tersebut.
~\cite{nasional:05:kurikulum}\footnote{Panduan Penyusunan Kurikulum Pendidikan Tinggi, Kemenristekdikti, 2016} Penyusunan kurikulum 2018 berpegang pada prinsip bahwa 
kurikulum yang baik adalah kurikulum yang tidak hanya kokoh, secara teoretis konseptual dapat dipertanggungjawabkan, namun juga secara praktis dapat dilaksanakan. Selain itu kurikulum juga harus cukup fleksibel agar dapat mengakomodasi perubahan-perubahan, namun tanpa kehilangan ciri atau kekhasan dari program studi. Dalam penyusunan kurikulum 2018 program studi Informatika secara khusus juga memperhatikan Kerangka Kualifikasi Nasional Indonesia (KKNI) yang tertuang dalam Peraturan Presiden no 8 tahun 2012. KKNI merupakan pernyataan kualitas SDM Indonesia, di mana tolok ukur kualifikasinya ditetapkan berdasarkan capaian pembelajaran \textit{(learning outcomes)} yang dimilikinya. Tahapan penyusunan kurikulum 2018 meliputi kegiatan sebagai berikut: 
\begin{enumerate}
\item Melakukan evaluasi diri dan pelacakan lulusan.
\item Merumuskan profil lulusan.
\item Menentukan capaian pembelajaran.
\item Menentukan bahan kajian.
\item Menyusun matriks pembelajaran dan bahan kajian.
\item Membentuk mata kuliah.
\item Menyusun struktur kurikulum dan menentukan metode pembelajaran.
\end{enumerate}

Teknologi baru sekarang memungkinkan untuk membangun layanan yang menjawab pertanyaan-pertanyaan secara otomatis. Sebagian besar data yang diperlukan untuk menjawab pertanyaan-pertanyaan dihasilkan oleh badan-badan publik. Namun, seringkali data yang diperlukan belum tersedia dalam bentuk yang mudah digunakan. Data terbuka berbicara tentang bagaimana membuka potensi dari informasi resmi dan lainnya untuk mengaktifkan layanan-layanan baru. Gagasan dari data terbuka itu sendiri bertujuan agar setiap orang bebas untuk mengakses dan menggunakan ulang untuk berbagai tujuan - sudah bergulir dalam beberapa tahun ini. Data terbuka itu sendiri memiliki arti yaitu data yang dapat secara bebas digunakan, digunakan ulang dan didistribusi ulang oleh siapapun - hanya patuh, umumnya, pada keharusan untuk menyebutkan siapa penciptanya dan berbagi dengan lisensi yang sama.\footnote{"Data Terbuka", \url{http://opendatahandbook.org/guide/id/what-is-open-data/}} Defini Terbuka memberikan rincian yang tepat apa yang dimaksud data terbuka. Ringkasannya adalah:
\begin{enumerate}
\item \textbf{Ketersediaan dan Akses:} data harus tersedia secara keseluruhan dan tidak lebih dari pada biaya reproduksi yang masuk akal, akan lebih baik bila bisa dilakukan dengan pengunduhan melalui internet.
\item \textbf{Penggunaan-ulang dan Distribusi ulang:} data harus disediakan di bawah ketentuan yang mengizinkan untuk penggunaan-upang dan pendistribusian ulang termasuk memadukan dengan kumpulan data lainnya.
\item \textbf{Partisipasi Universal:} setiap orang harus diperbolehkan untuk menggunakan, menggunakan-ulang dan mendistribusi ulang - tidak boleh ada diskriminasi terhadap bidang kerja atau perseorangan atau kelompok.
\end{enumerate}

Untuk menampung data terbuka dapat digunakan \textit{github} sebagai salah satu penampung untuk menyimpan data. \textit{Github} sebagai \textit{open source} di dalamnya dapat menyimpan data dalam \textit{format} \textit{JSON}. \textit{JSON} digunakan sebagai acuan dalam pembuatan pohon kurikulum 2018. \textit{Format JSON} bakal diubah ke dalam \textit{DOT Language} untuk menghasilkan graf. Penggunaan graf ditujukan agar mempermudah dalam melihat kurikulum baru. Untuk mem \textit{visualisasi} kan graf digunakan \textit{viz.js}, \textit{Viz.js} ini nantinya akan membantu dalam menghasilkan graf yang akan di tampilkan.

\section{Rumusan Masalah}
Berdasarkan latar belakang masalah yang telah dijelaskan, rumusan masalah pada penelitian ini adalah:
\begin{enumerate}
\item Bagaimana menerjemahkan perangkat lunak dalam bentuk \textit{word} ke bentuk \textit{JSON}.
\item Bagaimana membuat perangkat lunak dari bentuk \textit{JSON} ke dalam graf.
\end{enumerate} 


\section{Tujuan}
Berdasarkan rumusan masalah di atas, maka tujuan dari penelitian ini adalah: 
\begin{enumerate}
\item Membuat terjemahan dari bentuk \textit{word} ke dalam bentuk \textit{JSON}.
\item Membuat perangkat lunak dalam bentuk graf.
\end{enumerate}

\section{Batasan Masalah}
\label{sec:batasan}
Adapun batasan masalah yang didapat dari tujuan dan rumusan masalah di atas adalah:

\begin{enumerate}
\item Perangkat lunak menghasilkan pohon kurikulum.
\end{enumerate}

\section{Metodologi Penelitian}
\label{sec:Metodologi Penelitian}
Dalam penyusunan skripsi ini mengikuti langkah-langkah metodologi penelitian sebagai berikut:

\begin{enumerate}
\item Melakukan studi pustaka untuk dijadikan referensi dalam pembangunan perangkat lunak pohon kurikulum.
\item Melakukan studi tentang penggunaan vis.js untuk visualisasi pohon kurikulum.
\item Melakukan studi tentang data terbuka.
\item Melakukan studi tentang cara penggunaan DOT \textit{Language}
\end{enumerate}

\section{Sistematika Penulisan}
\label{sec:Sistematika Penulisan}
Keseluruhan bab yang disusun dalam penelitian ini terbagi kedalam bab-bab sebagai berikut:
\begin{enumerate}
\item Bab 1 Pendahuluan
Bab ini membahas mengenai latar belakang, rumusan masalah, tujuan, batasan masalah,
metodologi penelitian dan sistematika penulisan.
\item Bab 2 Dasar Teori
Bab ini membahas mengenai pengertian graf, data terbuka, JSON, apa itu DOT \textit{Language}, dan visualisasi menggunakan \textit{viz.js.}
\item Bab 3 Analisis 
Bab ini akan membahas mengenai JSON yang dapat dipakai sebagai sumber data terbuka.
\item Bab 4 Perancangan 
Bab ini akan membahas mengenai perancangan struktur pohon kurikulum untuk mahasiswa, di mana nanti di dalamnya akan berisi mata kuliah, syarat tempuh, dan syarat lulus. 
\item Bab 5 Implementasi dan Pengujian
Bab ini akan membahas mengenai pengujian, implementasi kode program untuk membuat pohon kurikulum.
\item Bab 6 Kesimpulan dan Saran
Bab ini akan membahas mengenai kesimpulan dari penelitian yang telah dilakukan dan
saran-saran untuk pengembangan lebih lanjut dari penelitian ini.
\end{enumerate}