\chapter{Kesimpulan Dan Saran}
\label{chap:Kesimpulan Dan Saran}

Bab ini berisi kesimpulan yang didapat dari pembangunan perangkat lunak serta saran-saran yang dapat digunakan
untuk pengembangan perangkat lunak selanjutnya.

\section{Kesimpulan}
\label{sec: Kesimpulan}

Setelah proses penelitian selesai dilakukan, dapat ditarik beberapa kesimpulan sebagai berikut:
\begin{enumerate}
\item data JSON sudah berhasil dibuat, contoh pemakaian dapat dilihat di Bab 3.1 dan di upload pada \textit{url} \url{https://github.com/ftisunpar/data}.
\item Telah berhasil dibangun sebuah perangkat lunak yaitu pohon kurikulum yang dapat digunakan untuk membantu menjabarkan mata kuliah di setiap semester pada prodi jurusan Teknik Informatika dan ditemukan kesalahan pada dokumen kurikulum 2018 versi 0.97 yaitu tidak ditemukannya mata kuliah yang menjadi syarat lulus pada mata kuliah metode optimisasi.
\end{enumerate}


\section{Saran}
\label{sec: Saran}

Berikut ini merupakan saran yang diharapkan dapat menjadi masukan apabila dikemudian hari hendak dilakukan pengembangan lebih lanjut terhadap perangkat lunak ini :

\begin{enumerate}
	\item Dapat memanfaatkan prasayarat.json yang telah disimpan di \url{https://github.com/ftisunpar/data} untuk pembuatan grafik pada kurikulum selanjutnya.

\end{enumerate}