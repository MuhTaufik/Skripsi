%versi 2 (8-10-2016)
\chapter{Dasar Teori}
\label{chap:Dasar Teori}
Pada bab ini akan diuraikan teori-teori yang berhubungan dengan pembangunan pohon kurikulum. Teori-teori tersebut adalah teori tentang pengertian graph, DOT language, dan visualisasi pohon menggunakan \textit{vis.js}.

\section{Graph}
\label{sec: Graph}

\subsection{Definisi Graph}
\label{sec: Definisi Graph}
Suatu graph didefinisikan oleh himpunan verteks dan himpunan sisi (edge).
Verteks menyatakan entitas-entitas data dan sisi menyatakan keterhubungan antara
verteks. Biasanya untuk suatu graph G digunakan notasi matematis. 
\begin{lstlisting}
G=(V,E)
G = Graph
V = Simpul atau vertex, atau node, atau titik
E = Sisi atau garis, atau Edge
\end{lstlisting}

V adalah himpunan \textit{verteks} dan E himpunan sisi yang terdefinisi antara pasangan-pasangan verteks. Sebuah sisi antara verteks x dan y ditulis {x, y}. Suatu graph H = (V1,E1) disebut subgraph dari graph G jika V1 adalah himpunan bagian dari V dan E1 himpunan bagian dari E.
\subsection{Istilah dalam Graph}
\label{sec: Istilah dalam Graph}
\begin{enumerate}
\item \textit{Incident}
Jika e merupakan busur dengan simpul-simpulnya adalah v dan w yang
ditulis e=(v,w), maka v dan w disebut "terletak" pada e, dan e disebut incident
dengan v dan w.
\item \textit{Degree}
Di dalam Graph ada yang disebut dengan \textit{Degree}, \textit{Degree} mempuyai 3 jenis antara lain :
\begin{itemize}
\item Degree dari suatu verteks x dalam undigraph adalah jumlah busur yang
incident dengan simpul tersebut.
\item Indegree dari suatu verteks x dalam digraph adalah jumlah busur yang
kepalanya incident dengan simpul tersebut, atau jumlah busur yang "masuk" atau menuju simpul tersebut.
\item Outdegree dari suatu verteks x dalam digraph adalah jumlah busur yang
ekornya incident dengan simpul tersebut, atau jumlah busur yang "keluar"
atau berasal dari simpul tersebut.
\end{itemize}
\item \textit{Adjacent}
Pada graph tidah berarah, 2 buah simpul disebut adjacent bila ada busur yang
menghubungkan kedua simpul tersebut. Simpul v dan w disebut adjacent. 
Pada graph berarah, simpul v disebut adjacent dengan simpul w bila ada busur
dari w ke v.
\item \textit{Successor dan Predecessor}
Pada graph berarah, bila simpul v \textit{adjacent} dengan simpul w, maka simpul v adalah \textit{successor} simpul w, dan simpul w adalah \textit{predecessor} dari simpul v.
\end{enumerate}

\section{DOT Language}
\label{secl : DOT Language}


\subsection{Dasar Menggambar Grafik}
\label{sec: Dasar Menggambar Grafik}


\subsection{Atribut Menggambar}
\label{sec: Atribut Menggambar}

\subsubsection{Bentuk dan Label}
\label{sec: Bentuk dan Label}

\subsubsection{Bentuk Grafik}
\label{sec: Bentuk Grafik}

\subsubsection{Menggambar Ukuran dan Jarak}
\label{sec: Menggambar Ukuran dan Jarak}

\subsubsection{Penempatan Sisi dan Simpul}
\label{sec: Penempatan Sisi dan Simpul}

\subsection{Node Ports}
\label{sec: Node Ports}

\subsubsection{Pengelompokan}
\label{sec: Pengelompokan}


\section{Visualisasi Graph dengan Viz.js}
\label{sec: Visualisasi Graph dengan Viz.js}

