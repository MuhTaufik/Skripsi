\documentclass[a4paper,twoside]{article}
\usepackage[T1]{fontenc}
\usepackage[bahasa]{babel}
\usepackage{graphicx}
\usepackage{graphics}
\usepackage{float}
\usepackage[cm]{fullpage}
\pagestyle{myheadings}
\usepackage{etoolbox}
\usepackage{setspace} 
\usepackage{lipsum} 
\setlength{\headsep}{30pt}
\usepackage[inner=2cm,outer=2.5cm,top=2.5cm,bottom=2cm]{geometry} %margin
% \pagestyle{empty}

\makeatletter
\renewcommand{\@maketitle} {\begin{center} {\LARGE \textbf{ \textsc{\@title}} \par} \bigskip {\large \textbf{\textsc{\@author}} }\end{center} }
\renewcommand{\thispagestyle}[1]{}
\markright{\textbf{\textsc{AIF401/AIF402 \textemdash Rencana Kerja Skripsi \textemdash Sem. Genap 2017/2018}}}

\onehalfspacing
 
\begin{document}

\title{\@judultopik}
\author{\nama \textendash \@npm} 

%tulis nama dan NPM anda di sini:
\newcommand{\nama}{Muhammad Taufik Adianto}
\newcommand{\@npm}{2012730089}
\newcommand{\@judultopik}{Pemodelan Kuliah Kurikulum 2018 Dalam Format JSON} % Judul/topik anda
\newcommand{\jumpemb}{1} % Jumlah pembimbing, 1 atau 2
\newcommand{\tanggal}{03/02/2018}
\maketitle

\pagenumbering{arabic}

\section{Deskripsi}
Kurikulum menjadi komponen acuan oleh setiap satuan pendidikan. Kurikulum berkembang sejalan dengan perkembangan teori dan praktek pendidikan, selain itu juga bervariasi sesuai dengan aliran atau teori pendidikan yang dianut pemangku kebijakan. Kurikulum memiliki kedudukan yang sangat sentral dalam keseluruhan proses pendidikan. Kurikulum juga mengarahkan segala bentuk aktivitas pendidikan kepada tercapainya tujuan-tujuan pendidikan. Sehingga kurikulum menjadi elemen pokok dalam sebuah layanan program pendidikan. Kurikulum juga memiliki peranan penting dalam pendidikan, kaitannya yaitu dengan penentuan arah, isi, dan proses pendidikan yang pada akhirnya menentukan macam dan kualifikasi lulusan suatu lembaga pendidikan. Dengan kata lain kurikulum menjadi syarat mutlak dari pendidikan dan kurikulum merupakan bagian yang tak terpisahkan dari pendidikan dan pengajaran. Sehingga sangatlah sulit dibayangkan bagaimana bentuk pelaksanaan suatu pendidikan tanpa adanya kurikulum.

Pada dasarnya kurikulum tidak hanya berisikan tentang petunjuk teknis materi pembelajaran. Kurikulum merupakan sebuah program terencana dan menyeluruh, yang secara tidak langsung menggambarkan manajemen pendidikan suatu bangsa. Dengan begitu otomatis kurikulum memegang peran yang sangat penting dan strategis dalam kemajuan dunia pendidikan suatu negara.

Efektifitas implementasi dan pengembangan kurikulum di lapangan sangatlah bergantung pada kompetensi sumber daya yang tersedia di universitas, untuk memfasilitasi pengajar dalam mengartikulasi topik-topik yang termuat dalam kurikulum. Pengajar yang menjalankan segala sesuatu yang terjadi dalam kelasnya. Sehingga keberhasilan pengembangan kurikulum juga bergantung pada manajemen dari setiap pengajar. Kurikulum sendiri pada setiap satuan pendidikan berfungsi sebagai alat penggerak pendidikan. Dengan kesesuaian dan ketepatan setiap komponen yang ada dalam kurikulum diharapkan sasaran dan tujuan pendidikan akan tercapai secara maksimal.

Dikarenakan peran kurikulum sendiri sangatlah penting dalam upaya pencapaian tujuan pendidikan nasional, maka pemerintah Indonesia telah melakukan berbagai macam upaya untuk merevisi, mengembangkan dan menyempurnakan desain kurikulum pendidikan nasional Indonesia untuk bisa menghasilkan proses dan produk pendidikan yang bermutu dan kompetitif. Kurikulum tidak bersifat statis, sehingga munculnya kurikulum disesuaikan dengan perkembangan zaman dan tuntutan kemajuan kehidupan dalam masyarakat.

Kurikulum memang selalu berkembang dan menyelaraskan diri dengan kemajuan zaman. Untuk itu  pengembangan kurikulum berupa proses yang dinamis dan integratif yang memang perlu diupayakan melalui langkah-langkah yang sistematis, profesional dan melibatkan seluruh aspek yang terkait dalam tercapainya tujuan pendidikan. Namun jika dilihat di lapangan perubahan kurikulum yang dirasa menjadi suatu siklus yang ekstrem malah menunjukkan banyak masalah karena perubahan kurikulum itu sendiri yang terlalu sering. Setiap pergantian kepemimpinan atau perubahan menteri pendidikan sendiri hampir bisa dipastikan akan terjadi perubahan kurikulum yang akhirnya membuat para aktor di bidang pendidikan tersesat di dalam kurikulum yang tidak jelas.

\section{Rumusan Masalah}
Berdasarkan latar belakang masalah yang telah dijelaskan, rumusan masalah pada penelitian ini adalah:
\begin{enumerate}
\item Tujuan diubahnya kurikulum lama (2013) ke kurikulum baru (2018).
\item Alasan diubahnya kurikulum lama (2013) ke kurikulum baru (2018).
\item Dampak yang terjadi dengan danya perubahan kurikulum.
\end{enumerate} 


\section{Tujuan}
Berdasarkan rumusan masalah di atas, maka tujuan dari penelitian ini adalah: 
\begin{enumerate}
\item Mahasiswa mengetahui tentang tujuan perubahan kurikulum.
\item Mahasiswa mengetahui tentang alasan diubahnya kurikulum lama ke kurikulum baru.
\item Mahasiswa mengetahui tentang dampak perubahan kurikulum.
\end{enumerate}


\section{Deskripsi Perangkat Lunak}
Untuk mencapai tujuan yang telah disebutkan di atas, maka perlu dibangun sebuah pohon kurikulum baru yang dapat menangani masalah yang telah disebutkan. Perangkat lunak akhir yang akan dibuat memiliki penggambaran minimal sebagai berikut:

\begin{itemize}
	\item Mahasiswa dapat melihat mata kuliah wajib yang ada di pohon kurikulum.
	\item Mahasiswa dapat melihat mata kuliah yang memiliki syarat tempuh.
	\item Mahasiswa dapat melihat mata kuliah yang memiliki syarat lulus.
	\item Mahasiswa dapat mengetahui standar sks tempuh per semester.
\end{itemize}

\section{Detail Pengerjaan Skripsi}
Bagian-bagian pekerjaan skripsi ini adalah sebagai berikut :
	\begin{enumerate}
		\item Melakukan pembelajaran mengenai kurikulum.
		\item Melakukan studi mengenai DOT Language.
		\item Melakukan studi visualisasi menggunakan viz.js.
		\item Melakukan analisis terhadap penggunaan pohon kurikulum pada kurikulum.
		\item Melakukan implementasi untuk melihat hasil pohon kurikulum.
		\item Menulis dokumen skripsi.
	\end{enumerate}

\section{Rencana Kerja}
Berikut ini adalah rencana untuk menyelesaikan skripsi. Rencana kerja dibagi menjadi dua bagian yaitu
yang akan dilakukan pada saat mengambil kuliah AIF401 Skripsi 1 dan pada saat mengambil kuliah AIF402
Skripsi 2.

\begin{center}
  \begin{tabular}{ | c | c | c | c | l |}
    \hline
    1*  & 2*(\%) & 3*(\%) & 4*(\%) &5*\\ \hline \hline
    1   &   &   &  &  \\ \hline
    2   &   &   &  &  \\ \hline
    3   &   &   &  & {\footnotesize }  \\ \hline
    4   &   &   &  & {\footnotesize } \\ \hline
    5   &   &   &  & {\footnotesize } \\ \hline
    6   &  &    &  & {\footnotesize Penulisan Dokumen SKripsi }\\ \hline
    
    Total  & 100  & 40  & 60 &  \\ \hline
                          \end{tabular}
\end{center}

Keterangan (*)\\
1 : Bagian pengerjaan Skripsi (nomor disesuaikan dengan detail pengerjaan di bagian 5)\\
2 : Persentase total \\
3 : Persentase yang akan diselesaikan di Skripsi 1 \\
4 : Persentase yang akan diselesaikan di Skripsi 2 \\
5 : Penjelasan singkat apa yang dilakukan di S1 (Skripsi 1) atau S2 (skripsi 2)

\vspace{1cm}
\centering Bandung, \tanggal\\
\vspace{2cm} \nama \\ 
\vspace{1cm}

Menyetujui, \\
\ifdefstring{\jumpemb}{2}{
\vspace{1.5cm}
\begin{centering} Menyetujui,\\ \end{centering} \vspace{0.75cm}
\begin{minipage}[b]{0.45\linewidth}
% \centering Bandung, \makebox[0.5cm]{\hrulefill}/\makebox[0.5cm]{\hrulefill}/2013 \\
\vspace{2cm} Nama: \makebox[3cm]{\hrulefill}\\ Pembimbing Utama
\end{minipage} \hspace{0.5cm}
\begin{minipage}[b]{0.45\linewidth}
% \centering Bandung, \makebox[0.5cm]{\hrulefill}/\makebox[0.5cm]{\hrulefill}/2013\\
\vspace{2cm} Nama: \makebox[3cm]{\hrulefill}\\ Pembimbing Pendamping
\end{minipage}
\vspace{0.5cm}
}{
% \centering Bandung, \makebox[0.5cm]{\hrulefill}/\makebox[0.5cm]{\hrulefill}/2013\\
\vspace{2cm} Nama: \makebox[3cm]{\hrulefill}\\ Pembimbing Tunggal
}

\end{document}

