\documentclass[a4paper,twoside]{article}
\usepackage[T1]{fontenc}
\usepackage[bahasa]{babel}
\usepackage{graphicx}
\usepackage{graphics}
\usepackage{float}
\usepackage[cm]{fullpage}
\pagestyle{myheadings}
\usepackage{etoolbox}
\usepackage{setspace} 
\usepackage{lipsum} 
\setlength{\headsep}{30pt}
\usepackage[inner=2cm,outer=2.5cm,top=2.5cm,bottom=2cm]{geometry} %margin
% \pagestyle{empty}

\makeatletter
\renewcommand{\@maketitle} {\begin{center} {\LARGE \textbf{ \textsc{\@title}} \par} \bigskip {\large \textbf{\textsc{\@author}} }\end{center} }
\renewcommand{\thispagestyle}[1]{}
\markright{\textbf{\textsc{AIF401/AIF402 \textemdash Rencana Kerja Skripsi \textemdash Sem. Genap 2017/2018}}}

\onehalfspacing
 
\begin{document}

\title{\@judultopik}
\author{\nama \textendash \@npm} 

%tulis nama dan NPM anda di sini:
\newcommand{\nama}{Muhammad Taufik Adianto}
\newcommand{\@npm}{2012730089}
\newcommand{\@judultopik}{Pemodelan Kuliah Kurikulum 2018 Dalam Format JSON} % Judul/topik anda
\newcommand{\jumpemb}{1} % Jumlah pembimbing, 1 atau 2
\newcommand{\tanggal}{13/02/2018}
\maketitle

\pagenumbering{arabic}

\section{Deskripsi}
Kurikulum didefinisikan sebagai seperangkat rencana dan pengaturan mengenai capaian pembelajaran 
lulusan, bahan kajian, proses, dan penilaian yang digunakan sebagai pedoman penyelenggaraan program studi menjadi sarana utama untuk mencapai tujuan tersebut. Penyusunan kurikulum 2018 berpegang pada prinsip bahwa 
kurikulum yang baik adalah kurikulum yang tidak hanya kokoh, secara teoretis konseptual dapat dipertanggungjawabkan, namun juga secara praktis dapat dilaksanakan. Selain itu kurikulum juga harus cukup fleksibel agar dapat mengakomodasi perubahan-perubahan, namun tanpa kehilangan ciri atau kekhasan dari program studi. Dalam penyusunan kurikulum 2018 program studi Informatika secara khusus juga memperhatikan Kerangka Kualifikasi Nasional Indonesia (KKNI) yang tertuang dalam Peraturan Presiden no 8 tahun 2012. KKNI merupakan pernyataan kualitas SDM Indonesia, di mana tolok ukur kualifikasinya ditetapkan berdasarkan capaian 
pembelajaran \textit{(learning outcomes)} yang dimilikinya. Tahapan penyusunan kurikulum 2018 meliputi kegiatan sebagai berikut: 
\begin{enumerate}
\item Melakukan evaluasi diri dan pelacakan lulusan.
\item Merumuskan profil lulusan.
\item Menentukan capaian pembelajaran.
\item Menentukan bahan kajian.
\item Menyusun matriks pembelajaran dan bahan kajian.
\item Membentuk mata kuliah.
\item Menyusun struktur kurikulum dan menentukan metode pembelajaran.
\end{enumerate}

Teknologi baru sekarang memungkinkan untuk membangun layanan yang menjawab pertanyaan-pertanyaan secara otomatis. Sebagian besar data yang diperlukan untuk menjawab pertanyaan-pertanyaan dihasilkan oleh badan-badan publik. Namun, seringkali data yang diperlukan belum tersedia dalam bentuk yang mudah digunakan. Data terbuka berbicara tentang bagaimana membuka potensi dari informasi resmi dan lainnya untuk mengaktifkan layanan-layanan baru. Gagasan dari data terbuka itu sendiri bertujuan agar setiap orang bebas untuk mengakses dan menggunakan ulang untuk berbagai tujuan - sudah bergulir dalam beberapa tahun ini. Data terbuka itu sendiri memiliki arti yaitu data yang dapat secara bebas digunakan, digunakan ulang dan didistribusi ulang oleh siapapun - hanya patuh, umumnya, pada keharusan untuk menyebutkan siapa penciptanya dan berbagi dengan lisensi yang sama. Defini Terbuka memberikan rincian yang tepat apa yang dimaksud data terbuka. Ringkasannya adalah:
\begin{enumerate}
\item \textbf{Ketersediaan dan Akses:} data harus tersedia secara keseluruhan dan tidak lebih dari pada biaya reproduksi yang masuk akal, akan lebih baik bila bisa dilakukan dengan pengunduhan melalui internet.
\item \textbf{Penggunaan-ulang dan Distribusi ulang:} data harus disediakan di bawah ketentuan yang mengizinkan untuk penggunaan-upang dan pendistribusian ulang termasuk memadukan dengan kumpulan data lainnya.
\item \textbf{Partisipasi Universal:} setiap orang harus diperbolehkan untuk menggunakan, menggunakan-ulang dan mendistribusi ulang - tidak boleh ada diskriminasi terhadap bidang kerja atau perseorangan atau kelompok.
\end{enumerate}

Untuk menampung data terbuka dapat digunakan \textit{github} sebagai salah satu penampung untuk menyimpan data. \textit{Github} sebagai \textit{open source} di dalamnya dapat menyimpan data dalam \textit{format} \textit{JSON}. \textit{JSON} digunakan sebagai acuan dalam pembuatan pohon kurikulum 2018. \textit{Format JSON} bakal diubah ke dalam \textit{DOT Language} untuk menghasilkan graf. Penggunaan graf ditujukan agar mempermudah dalam melihat kurikulum baru. Untuk mem \textit{visualisasi} kan graf digunakan \textit{viz.js}, \textit{Viz.js} ini nantinya akan membantu dalam menghasilkan graf yang akan di tampilkan.
\section{Rumusan Masalah}
Berdasarkan latar belakang masalah yang telah dijelaskan, rumusan masalah pada penelitian ini adalah:
\begin{enumerate}
\item Bagaimana menerjemahkan perangkat lunak dalam bentuk \textit{word} ke bentuk \textit{JSON}.
\item Bagaimana membuat perangkat lunak dari bentuk \textit{JSON} ke dalam graf.
\end{enumerate} 


\section{Tujuan}
Berdasarkan rumusan masalah di atas, maka tujuan dari penelitian ini adalah: 
\begin{enumerate}
\item Membuat terjemahan dari bentuk \textit{word} ke dalam bentuk \textit{JSON}.
\item Membuat perangkat lunak dalam bentuk graf.
\end{enumerate}


\section{Deskripsi Perangkat Lunak}
Untuk mencapai tujuan yang telah disebutkan di atas, maka perlu dibangun sebuah perangkat lunak baru yang dapat menangani masalah yang telah disebutkan. Perangkat lunak akhir yang akan dibuat memiliki penggambaran minimal sebagai berikut:

\begin{itemize}
	\item Perangkat lunak berbentuk pohon kurikulum.
	\item Perangkat lunak memperlihatkan mata kuliah yang mempunyai syarat lulus mata kuliah.	
	\item Perangkat lunak memperlihatkan mata kuliah yang mempunyai syarat tempuh mata kuliah.
\end{itemize}

\section{Detail Pengerjaan Skripsi}
Bagian-bagian pekerjaan skripsi ini adalah sebagai berikut :
	\begin{enumerate}
		\item Melakukan pembelajaran mengenai kurikulum 2018.
		\item Melakukan studi mengenai DOT Language.
		\item Melakukan studi visualisasi menggunakan viz.js.
		\item Melakukan analisis pembuatan pohon kurikulum.
		\item Melakukan perancangan terhadap perangkat lunak.
		\item Melakukan implementasi untuk melihat hasil pohon kurikulum.
		\item Menulis dokumen skripsi.
	\end{enumerate}

\section{Rencana Kerja}
Berikut ini adalah rencana untuk menyelesaikan skripsi. Rencana kerja dibagi menjadi dua bagian yaitu
yang akan dilakukan pada saat mengambil kuliah AIF401 Skripsi 1 dan pada saat mengambil kuliah AIF402
Skripsi 2.

\begin{center}
  \begin{tabular}{ | c | c | c | c | l |}
    \hline
    1*  & 2*(\%) & 3*(\%) & 4*(\%) &5*\\ \hline \hline
    1   &   &   & 5 &  \\ \hline
    2   &   &   & 10 &  \\ \hline
    3   &   &   & 10 &  \\ \hline
    4   &   &   & 15 &  \\ \hline
    5   &   &   & 20 & {\footnotesize Perancangan perangkat lunak} \\ \hline
    6   &  &    & 20 & {\footnotesize Implentasi visualisasi menggunakan viz.js}\\ \hline
    7   &   &   & 20 & {\footnotesize Menulis dokumen skripsi} \\ \hline
    
    Total  & 100  & 0  & 100 &  \\ \hline
                          \end{tabular}
\end{center}

Keterangan (*)\\
1 : Bagian pengerjaan Skripsi (nomor disesuaikan dengan detail pengerjaan di bagian 5)\\
2 : Persentase total \\
3 : Persentase yang akan diselesaikan di Skripsi 1 \\
4 : Persentase yang akan diselesaikan di Skripsi 2 \\
5 : Penjelasan singkat apa yang dilakukan di S1 (Skripsi 1) atau S2 (skripsi 2)

\vspace{1cm}
\centering Bandung, \tanggal\\
\vspace{2cm} \nama \\ 
\vspace{1cm}

Menyetujui, \\
\ifdefstring{\jumpemb}{2}{
\vspace{1.5cm}
\begin{centering} Menyetujui,\\ \end{centering} \vspace{0.75cm}
\begin{minipage}[b]{0.45\linewidth}
% \centering Bandung, \makebox[0.5cm]{\hrulefill}/\makebox[0.5cm]{\hrulefill}/2013 \\
\vspace{2cm} Nama: \makebox[3cm]{\hrulefill}\\ Pembimbing Utama
\end{minipage} \hspace{0.5cm}
\begin{minipage}[b]{0.45\linewidth}
% \centering Bandung, \makebox[0.5cm]{\hrulefill}/\makebox[0.5cm]{\hrulefill}/2013\\
\vspace{2cm} Nama: \makebox[3cm]{\hrulefill}\\ Pembimbing Pendamping
\end{minipage}
\vspace{0.5cm}
}{
% \centering Bandung, \makebox[0.5cm]{\hrulefill}/\makebox[0.5cm]{\hrulefill}/2013\\
\vspace{2cm} Nama: \makebox[3cm]{\hrulefill}\\ Pembimbing Tunggal
}

\end{document}

