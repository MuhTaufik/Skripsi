\chapter{Perancangan}
\label{chap: Perancangan}

Pada bab ini akan dibahas mengenai perancangan perangkat lunak yang diimplementasi pada pohon kurikulum.

\section{Kebutuhan Masukan dan Pengeluaran}
\label{sec: Kebutuhan Masukan dan Pengeluaran}
Perancangan perangkat lunak pohon kurikulum dengan membangkitkan dari JSON ke \textit{DOT}. Masukan perangkat lunak merupakan kode, nama, prasyarat, sks, semester, dan wajib.
Kebutuhan input dan output perangkat lunak:
\begin{itemize}
\item Masukan \\
Kebutuhan \textit{input} pada pohon kurikulum adalah
\begin{enumerate}
\item \textbf{semester}, berisikan urutan semester 1 sampai semester 8.
\item \textbf{kode}, berisikan kode mata kuliah.
\item \textbf{nama}, berisikan nama mata kuliah.
\item \textbf{sks}, memberitahukan kepada mahasiswa mata kuliah yang akan diambil memiliki beban berapa banyak.
\item \textbf{prasyarat}, isinya syarat tempuh atau lulus dari setiap mata kuliah.
\end{enumerate}
\item Keluaran\\
Output dari perangkat lunak adalah pohon kurikulum yang visualisasinya menggunakan \textit{viz.js}. Pohon kurikulum akan menampilkan node semester 1 sampai semester 8. Kemudian di setiap semesternya terdapat node yang berisikan kode, sks, dan nama matakuliah. Matakuliah yang ditampilka berupa matakuliah.
\end{itemize}

\section{Perancangan Perangkat Lunak Pohon Kurikulum}
\label{sec: Perancangan Perangkat Lunak Pohon Kurikulum}
Berikut rancangan pembuatan perangkat lunak pohon kurikulum:

\begin{enumerate}
\item Pemanggilan \textit{library}\\
Untuk membuat pohon kurikulum dibutuhkan \textit{library} sebagai bantuan untuk memanggil fungsi yang akan dijalankan. Ada tiga \textit{library} yang digunakan, yaitu \textit{viz.js, http, dan axios}. Masing - masing \textit{library} memiliki fungsi,
\begin{itemize}
\item \textit{viz.js}, fungsinya sebagai visualisasi dalam bentuk grafik.
\item \textit{http}, fungsinya menjalankan server web tanpa menggunakan program server web seperti \textit{Apache}. 
\item \textit{axios}, \textit{library} untuk http req, karena untuk mengakses data \textit{raw} di \textit{github} perlu \textit{request} data melalui \textit{http}.
\end{itemize} 
\item Membuat server untuk memanggil server yang berjalan di atas \textit{node.js}. Dengan cara memanggil \textit{createServer()} untuk membuat server HTTP. Di dalam \textit{createServer()} ada method get untuk memanggil \textit{url} yang berasal dari \textit{github}. Lalu di pnggil sebuah fungsi yang di dalamnya akan mengambil data yang berupa node isinya ada \textit{rankSep(), nodesMatkul(), dan edgesMatkul}.
\item Membuat \textit{rankSep()}, fungsinya untuk memanggil node setiap semester dan kode mata kuliah di setiap semester dan menampilkannya sesuai semester yang ada.
\item Membuat \textit{nodesMatkul()}, fungsinya memanggil kode mata kuliah, sks dan nama mata kuliah sesuai dengan semester yang ada. 
\item Membuat \textit{edgesMatkul()}, fungsinya untuk membuat arah setiap pada \textit{node} satu ke \textit{node} lainnya. \textit{Edge} dibedakan menjadi dua ada yang garis lurus dan ada yang garis putus - putus. Bedanya jika garis lurus maka mata kuliah tertentu memiliki syarat lulus sedangkan garis putus - putus menandakan bahwa syaratnya tempuh.

\end{enumerate}


\section{Perancangan Antarmuka Membuat Pohon Kurikulum}
\label{sec: Perancangan Antarmuka Membuat Pohon Kurikulum}
Untuk memenuhi kebutuhan interaksi antara pengguna dengan perangkat lunak, maka dirancanglah sebuah antarmuka berupa pohon kurikulum. Rancangan antarmuka dibuat dengan visualsasi menggunakan \textit{viz.js}. 