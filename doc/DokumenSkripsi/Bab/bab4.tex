\chapter{Perancangan}
\label{chap: Perancangan}

Pada bab ini akan dibahas mengenai perancangan perangkat lunak yang diimplementasi pada pohon kurikulum.

\section{Kebutuhan Masukan dan Keluaran}
\label{sec: Kebutuhan Masukan dan Keluaran}
Pada perancangan perangkat lunak pohon kurikulum dilakukan dengan membangkitkan dari JSON ke \textit{DOT}. Kebutuhan masukan dan keluaran perangkat lunak sebagai berikut:
\begin{itemize}
\item Masukan \\
\begin{enumerate}
\item \textbf{data JSON}, berisikan pernyataan JSON yang dipakai sebagai acuan dalam membangkitkan \textit{DOT}. data JSON dapat dilihat di lampiran B atau pada bab 3.1.
\item \textbf{semester}, berisikan urutan semester 1 sampai semester 8.
\item \textbf{kode}, berisikan kode mata kuliah.
\item \textbf{nama}, berisikan nama mata kuliah.
\item \textbf{sks}, memberitahukan kepada mahasiswa mata kuliah yang akan diambil memiliki beban berapa banyak.
\item \textbf{prasyarat}, isinya syarat tempuh atau lulus dari setiap mata kuliah.
\end{enumerate}
\item Keluaran\\
Keluaran dari perangkat lunak adalah pohon kurikulum yang visualisasinya menggunakan \textit{viz.js}. Pohon kurikulum akan menampilkan node semester 1 sampai semester 8. Kemudian di setiap semesternya terdapat node yang berisikan kode, sks, dan nama matakuliah. Matakuliah yang ditampilka berupa matakuliah.
\end{itemize}

\section{Perancangan Perangkat Lunak Pohon Kurikulum}
\label{sec: Perancangan Perangkat Lunak Pohon Kurikulum}
Berikut rancangan pembuatan perangkat lunak pohon kurikulum:

\begin{enumerate}
\item Memanggil \textit{library}\\
Untuk membuat pohon kurikulum dibutuhkan \textit{library} sebagai bantuan untuk memanggil fungsi yang akan dijalankan. Ada tiga \textit{library} yang digunakan, yaitu \textit{viz.js, http, dan axios}. Masing - masing \textit{library} memiliki fungsi,
\begin{itemize}
\item \textit{viz.js}, fungsinya sebagai visualisasi dalam bentuk grafik.
\item \textit{http}, fungsinya menjalankan \textit{server} di \textit{web} tanpa menggunakan program server web seperti \textit{Apache}. 
\item \textit{axios}, \textit{library} untuk \textit{http request}, karena untuk mengakses data \textit{raw} di \textit{github} perlu \textit{request} data melalui \textit{http}.
\end{itemize} 
\item Membuat server, untuk memanggil server yang berjalan di atas \textit{node.js}. Caranya dengan memanggil \textit{createServer()} untuk membuat server HTTP. Di dalam \textit{createServer()} ada method get untuk memanggil \textit{url} yang berasal dari \textit{github}. Lalu di pnggil sebuah fungsi yang di dalamnya akan mengambil data yang berupa node isinya ada \textit{rankSep(), nodesMatkul(), dan edgesMatkul}.
\item Membuat \textit{rankSep()}, fungsinya untuk memanggil node setiap semester dan kode mata kuliah di setiap semester dan menampilkannya sesuai semester yang ada.
\item Membuat \textit{nodesMatkul()}, fungsinya memanggil kode mata kuliah, sks dan nama mata kuliah sesuai dengan semester yang ada. 
\item Membuat \textit{edgesMatkul()}, fungsinya untuk membuat arah setiap pada \textit{node} satu ke \textit{node} lainnya. \textit{Edge} dibedakan menjadi dua ada yang garis lurus dan ada yang garis putus - putus. Bedanya jika garis lurus maka mata kuliah tertentu memiliki syarat lulus sedangkan garis putus - putus menandakan bahwa syaratnya tempuh.

\end{enumerate}


\section{Perancangan Antarmuka Pohon Kurikulum}
\label{sec: Perancangan Antarmuka Pohon Kurikulum}
Untuk memenuhi kebutuhan interaksi antara pengguna dengan perangkat lunak, maka dirancanglah sebuah antarmuka berupa pohon kurikulum. Rancangan antarmuka dibuat dengan cara membangkitkan menggunakan \textit{viz.js}. Setelah itu pada saat pemanggilan antarmuka dapat diatur bentuk dan keluaran yang akan dipakai untuk membangkitkan pohon kurikulum. Beberapa opsi parameter yang dapat digunakan untuk merubah tampilan dari graf yang akan ditampilkan, yaitu dengan menggunakan \textit{engine} yang dapat mengatur mesin \textit{Graphviz} untuk digunakan, salah satunya "circo", "dot", "fdp", "neato", "osage", or "twopi". Selain \textit{engine} ada juga parameter pilihan. parameter ini mengatur apakah mata kuliah pilihan akan ditampilkan atau sebaliknya. Cara pemanggilan pilihan sebagai berikut:
\begin{lstlisting}
...... PILIHAN=false
\end{lstlisting}
Jika dituliskan seperti itu maka akan menghasilkan pohon kurikulum yang berisikan semester, kode, nama mata kuliah, sks, dan mata kuliah wajib saja sementara mata kuliah pilihan tidak ditampilkan

