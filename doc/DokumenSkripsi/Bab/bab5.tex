\chapter{Implementasi dan Pengujian Perangkat Lunak}
\label{chap: Implementasi dan Pengujian Perangkat Lunak}

Bab ini terdiri atas dua bagian, yaitu Implementasi Perangkat Lunak dan Pengujian Perangkat Lunak. Bagian implementasi berisi penjelasan lingkungan pengembangan perangkat lunak dan hasil implementasi. Sedangkan bagian pengujian berisi hasil pengujian fungsional dan eksperimental terhadap perangkat lunak yang telah dibangun.

\section{Implementasi Perangkat Lunak}
\label{sec: Implementasi Perangkat Lunak}

Pada bagian ini akan dibahas mengenai implementasi perangkat lunak yang telah dibangun. Sub bab ini terdiri atas tiga bagian, yaitu lingkungan perangkat lunak, hasil implementasi perangkat lunak, dan Pengujian fungsional.

\subsection{Lingkungan Implementasi Perangkat Lunak}
\label{sec: Lingkungan Implementasi Perangkat Lunak}

Dalam proses membangun perangkat lunak ini digunakan spesifikasi perangkat sebagai berikut:

\begin{enumerate}
\item Lingkungan Pembangunan \\
\begin{itemize}
\item Processor : Intel® Core™i7-4702MQ 2.2-3.2GHz
\item RAM : 4.00 GB
\item Harddisk : 1TB
\item VGA : NVIDIA GeForce GT 740M
\item Sistem Operasi Komputer : Windows 10 Education 64-bit
\end{itemize}

\item Lingkungan Implementasi
\begin{itemize}
\item Sistem Operasi Server : \textit{Node.js}
\item Tools : \textit{Visual Studio Code}
\item Bahasa Pemrograman : Javascript
\item Framework : \textit{viz.js}
\end{itemize}

\end{enumerate}

\subsection{Hasil Implementasi}
\label{sec: Hasil Implementasi}

